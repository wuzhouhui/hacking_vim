% vim: ts=4 sts=4 sw=4 et tw=75
\chapter{前言}
\label{chap:preface}
\marginpar{1}
在计算机发展的早期, 系统资源非常有限, 开发人员必须想尽办法优化他们的应
用程序, 文本编辑器也是如此. Vim 是当时最流行的编辑器之一, 因为系统资源
方面的限制, 它被优化得近乎完美.

从那以后, 计算机技术得到了快速的发展, 虽然系统资源已经没有以前那么紧张
了, 但是 Vim 仍然坚持最初的原则.

乍看起来, Vim 好像并没有什么值得称道的地方, 但是, 如果你能透过它简单
的用户界面看到其本质, 就会明白为什么到了如今这个年代, Vim 仍然是众多
用户最喜爱的编辑器.

它几乎囊括了你想要的任何一个功能, 即使有所遗漏, 也可以通过插件或脚本
来实现. 正是由于其出色的灵活性, 使得它成为了众多任务的理想工具, 以及
世界上最先进的编辑器之一.

每天都有大量的新用户加入 Vim 社区, 并开始用 Vim 处理他们的日常工作.
虽然有时候 Vim 使用起来比较复杂, 但是与其他编辑器相比, 人们还是更愿意
选择 Vim, 本书就是为这些用户而写.

通过阅读本书, 用户可以更加得心应手地使用 Vim, 从而提高工作效率. 他们得
到的不仅仅是一个优化的编辑器, 还有优化的工作流程. 本书帮助用户更加自如
地使用 Vim, 从把它当作一个简单的文本编辑器开始, 一直到其他日常工作, Vim 
都可以胜任.

祝你阅读愉快!

\section*{本书主要讨论什么}
\marginpar{2}
第 \ref{chap:getting_started_with_vim} 章: 入门, 介绍 Vim 和它的几个
比较有名的亲戚, 并简要介绍它们和 vi 的关系及其历史.

第 \ref{chap:personalizing_vim} 章: 个性化 Vim, 介绍如何配置 Vim, 使得 
它更符合用户的个人需求. 主要介绍如何修改字体, 配色方案, 状态行, 菜单,
与工具条.

第 \ref{chap:Better_navigation} 章: 快速导航, 介绍一些在多个文件中快速
导航的方法, 通过这些方法, 读者可以轻松地在多个文件之间穿梭.

第 \ref{chap:production_boosters} 章: 助推器, 介绍 Vim 的若干特性, 
描述了模版, 自动补全, 拆叠, 会话, 和寄存器的使用方法.

第 \ref{chap:advanced_formatting} 章: 讨论如何对文本和代码进行格式化.
它还介绍了如何使用外部工具, 使得 Vim 更加完美.

第 \ref{chap:basic_vim_scripting} 章: Vim 脚本基础, 这一章是为那些想学习
如何通过脚本来扩展 Vim 功能的人而准备的, 它介绍了脚本的基础知识, 经过
这一章的学习, 读者应该能够写出一些简单的脚本.

第 \ref{chap:extended_vim_scripting}: Vim 脚本进阶, 在第
\ref{chap:basic_vim_scripting} 的基础上, 再继续讲一些高级的脚本知识,
包括如何在 Vim 脚本中使用外部的脚本语言.

附录 \ref{chap:vim_can_do_everything}: 无所不能的 Vim, 提供了一张游戏
列表, 它们都是用 vim 脚本开发的, 这一章还简单讨论了如何使用脚本来实现
聊天和邮件工具, 另外还介绍了如何将 Vim 作为一个 IDE 来使用.

附录 \ref{chap:vim_configuration_alternatives}: Vim 配置管理, 展示如何
通过在线副本来更好地管理和获取 Vim 配置文件.

\section*{为了阅读本书, 你还需要什么}
最近十年, Vim 已经发展成一个功能非常丰富的编辑器, 这同时意味着最新版的某些
功能, 旧版可能并不支持.

Vim 已经移植到了多种平台中, 但并不是所有的功能对任意一种平台来说都是可用
的. 这主要是因为有些功能使用了和操作系统密切相关的特性, 而这些特性在其他
平台中可能并不提供.
