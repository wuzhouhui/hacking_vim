% vim: ts=4 sts=4 sw=4 tw=80
\chapter{高级 Vim 脚本编程}
\label{chap:extended_vim_scripting}
\marginpar{175}

在前面一章, 我们已经学习了开发 Vim 脚本的基础知识, 现在, 我们将要把前面所学的
知识融会贯通, 按照结构化的方法把它们组织起来, 并对脚本进行测试.

这一章涵盖的主题包括:
\begin{itemize}
    \item 如何组织 Vim 脚本的结构
    \item Vim 脚本开发的一些技巧
    \item 如何调试 Vim 脚本
    \item 如何在 Vim 脚本中使用其他脚本语言
\end{itemize}

阅读完这一章之后, 读者将有能力运用 Vim 脚本语言与其他脚本语言开发出自己的脚本.
也就是说, 读者将有能力扩展 Vim 的功能.

\section{脚本结构}
\label{sec:script_structure}

前面我们已经介绍了 Vim 脚本的各个要素, 现在我们需要知道如何把它们组织在一起,
从而形成一个完成的脚本.

在大部分情况下, Vim 脚本仅由单个文件组成, 因此这一章的示例也仅限于单个文件. 我
们还打算让其他人能够获取到脚本, 因此需要保证代码的可读性.

在下面的几节里, 我们将会逐个介绍一个格式良好的脚本的各个要素.
\marginpar{176}
