% vim: ts=4 sts=4 sw=4 et tw=75
\chapter{开始}
\label{chap:getting_started_with_vim}

\marginpar{7}
Vim (Vi IMproved) 编辑器最早由 Bram Moolenaar 于 1991 年 11 月发布,
当时只是作为 Unix vi 编辑器的 Amiga 平台克隆版.

一年后, Unix 平台的 Vim 发布, 之后, 它迅速成为了 vi 的替代版本.

由于宽松的授权和丰富的功能, 在开源社区的帮助下, Vim 逐渐流行起来.
越来越多的 Linux 发行版开始用 Vim 替换掉 vi. 虽然许多用户认为他们使用
的是 vi (如果他们是通过执行命令 \vi 来打开编辑的话), 可实际上
打开的是 Vim (命令 \vi 已经被 \vim 的链接替换掉, 所以经常会有人误以为
vi 和 Vim 是同一个程序).

在九十年代后期, vi 在编辑器之战中所输掉的劣势, 重新又被 Vim 给赢了回来,
编辑器之战指的是 vi 和 Emacs 之间的斗争. Bram 为 Vim 扩充了许多新特性, 
而这些特性原本被  Emacs 党利用, 作为论证 vim/vi 不如 Emacs 的论据, 即
使如此, Bram 仍然没有忘记当初人们开发 vi 的初衷.

如今, Vim 已经是一个功能丰富, 定制性强, 受人欢迎的编辑器. 它支持超过 200 
种语言的语法高亮, 自动补全, 折叠, 撤消/重做, 多重缓冲区/窗口/标签, 以及
其他特性.

本章主要介绍
\begin{itemize}
    \item 如何获取与安装 Vim 编辑器
    \item Vim 编辑器家族
    \item Vim 的发布许可证 
    \item 本书使用的公共术语
\end{itemize}
现在, 让我们正式开始本书的阅读过程.

\marginpar{8}
