% vim: ts=4 sts=4 sw=4 tw=80
\chapter{Vim 配置管理}
\label{chap:vim_configuration_alternatives}
\marginpar{215}

在第 \ref{chap:personalizing_vim} 章, 我们介绍了 Vim 的主要配置文件, 随着
阅读的深入, 我们不断地往 \texttt{vimrc} 中添加新的配置信息, 最终, 配置文件
可能会变得非常混乱, 难以管理.

在这个附录中, 我们将介绍一些组织 \texttt{vimrc} 的方法, 从小技巧一直到整个
配置系统.

最后, 我们将会介绍如何在多个不同的计算机中使用同一个 \texttt{vimrc} 文件,
方法是在网络中保存一份副本.

\section{保持 vimrc 整洁的技巧}
\label{sec:tips_for_keeping_your_vimrc_file_clean}

\texttt{vimrc} 是用户设置 Vim 的核心文件, 如果没有它, 就只能使用系统中原有
的设置, 因此我们要时刻保持 \texttt{vimrc} 的整洁, 并及时更新, 只有这样, 用
户才能时刻知道文件中包含了哪些内容. 有时候, 用户可能没办法在文件中找到自己
想要的内容, 笔者就曾经遇到过这种情况, 当时我的 \texttt{vimrc} 超过了 2000 行,
到那时我才意识到整洁的必要性. 保持 \texttt{vimrc} 整洁并组织良好的技巧有:
\marginpar{216}
\begin{enumerate}
	\item 保持 Vim 处于非兼容模式

	这个技巧可能并不会让 \texttt{vimrc} 更整洁, 至少不能马上看出来, 然而
	它却非常重要. 让 Vim 处于非兼容模式下就可以打开许多特性, 而这些特性
	会被很多技巧和脚本使用到. 所以, 最好在 \texttt{vimrc} 的第一行总是写上
	\texttt{set nocompatible}.
	
	\item 使用注释

	有时候, 我们会在 Vim 中改变一些设置, 然后再把设置添加到 \texttt{vimrc},
	过一段时间后, 当我们想要清理 \texttt{vimrc} 时, 就会发现我们已经记不住
	某段脚本是干嘛的, 以及为什么要添加它, 甚至连代码是哪儿来的都已经记不清
	了. 为了防止这种情况发生, 那就给新加的东西写上注释. 笔者建议在注释中包含这
	些内容: 代码的作用 (描述), 从哪儿得到的 (来源), 代码的原作者. 有了注释的
	帮助, 用户就能够方便地查询到代码的来源, 以及决定是否需要删除它们. 注释
	用引号开始, 比如 \texttt{"This is a comment}.

	\item 数据分组

	脚本通常需要一些额外的设置, 或者是用户想要为脚本中的某些功能设置一些额外
	的按键绑定 (映射). 为了能更方便地看出设置的归属, 最好为数据分组. 分组的
	条件有很多种, 笔者推荐以下这些 (按照在文件中出现的顺序, 从上到下排列):
	\begin{itemize}
	  \item 通用的全局设置
	  \item 自己私有的按键映射
	  \item 特定于脚本的设置, 按脚本分组
	  \item 其他设置
	\end{itemize}

	\item 使用多个文件
	
	有时候, \texttt{vimrc} 可能会变得非常巨大, 无论怎么组织都显得非常混乱. 对
	于这种情况, 最好把它切分成多个文件. 为了切分 \texttt{vimrc}, 首先把需要分
	出的内容剪切到另一个文件中, 文件名最好能描述出文件的作用, 然后在
	\texttt{vimrc} 中原来的位置用 \texttt{source} 命令
\end{enumerate}
